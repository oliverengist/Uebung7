\documentclass[10pt]{article}
\usepackage[ngerman]{babel} 
\usepackage{float}
\title{CS102 \LaTeX \thinspace \"Ubung}
\author{Oliver Engist}
\begin{document}
\maketitle
\section{Das ist der erste Abschnitt}
Irgend ein Text.
\section{Tabelle}
Unsere wichtigsten Daten finden Sie in  Tabelle \ref{table:Tabelle 1} .
\begin{table}[H]
\centering
\begin{tabular}{c|c|c|c}
& Punkte erhalten & Punkte möglich & \% \\
\hline
Aufgabe 1 & 10 & 10 & 100\\
Aufgabe 2 & 8 & 10 & 80\\
Aufgabe 3 & 12 & 18 & 66.7\\

\end{tabular}
\caption{Diese Tabelle kann auch andere Werte beinhalten.}
\label{table:Tabelle 1}
\end{table}
\section{Formeln}
\subsection{Pythagoras}
Der Satz des Pythagoras errechnet sich wie folgt: $a^2+b^2=c^2$ Daraus können wir die Länge der Hypothenuse wie folgt berechnen: $c=\sqrt{a^2+b^2}$.
\subsection{Summen}
Wir können auch die Formel für eine Summe angeben: \begin{equation}
s=\sum_{i=0}^n i = \frac{n*(n+1)}{2}
\end{equation}

\end{document}
